\instituicao[a]{Universidade Federal de Santa Catarina} % Opcional
\departamento[a]{Departamento de Combate ao Crime}
\programa[o]{Programa de Treinamento de Jovens Talentos} 
\curso{Engenharia de Teias}
\documento[a]{Dissertação} % [o] para dissertação e trabalho de conclusão de curso [a] para tese
\grau{mestrado} % doutor, mestre, etc.
\titulo{Design de Teias: Como pegar bandidos usando teias}
\subtitulo{} % Opcional
\autor{Parker, Peter}
\local{Florianópolis} % Opcional (Florianópolis é o padrão)
\data{29}{Abril}{2017}
\orientador[Universidade ...]{Prof. Dr. Gunther Octavius}
\coorientador[Universidade ...]{Prof. Dr. ...}
\coordenador[Universidade ...]{Prof. Dr. ...}
\orientadornabanca{sim} % Se faz parte da banca definir como sim
\coorientadornabanca{nao} % Se faz parte da banca definir como sim
\bancaMembroA{Primeiro membro\\Universidade ...}  %Nome do presidente da banca
\bancaMembroB{Segundo membro\\Universidade ...}   % Nome do membro da Banca
\bancaMembroC{Terceiro membro\\Universidade ... \\(Videoconferência)} % Nome do membro da Banca
% \bancaMembroD{Quarto membro\\Universidade ... } % Nome do membro da Banca
% \bancaMembroE{Quinto membro\\Universidade ...}  % Nome do membro da Banca
% \bancaMembroF{Sexto membro\\Universidade ...}   % Nome do membro da Banca
% \bancaMembroG{Sétimo membro\\Universidade ...}  % Nome do membro da Banca

\dedicatoria{Este trabalho é dedicado aos meus colegas de classe e aos meus queridos pais.}

\agradecimento{Inserir os agradecimentos aos colaboradores à execução do trabalho. Inserir os agradecimentos aos colaboradores à execução do trabalho.}

\epigrafe{Texto da Epígrafe. Citação relativa ao tema do trabalho. É opcional. A epígrafe pode também aparecer na abertura de cada seção ou capítulo.}
{(Autor da epígrafe, ano)}

\textoresumo {O texto do resumo deve ser digitado, em um único bloco, sem espaço de parágrafo. O resumo deve ser significativo, composto de uma sequência de frases concisas, afirmativas e não de uma enumeração de tópicos. Não deve conter citações. Deve usar o verbo na voz passiva. Abaixo do resumo, deve-se informar as palavras-chave (palavras ou expressões significativas retiradas do texto) ou, termos retirados de thesaurus da área.}
\palavraschave{Palavra-chave 1. Palavra-chave 2.  Palavra-chave 3. }

\textoresumoexpandido{
\tituloresumoexpandido{Introdução}
O resumo expandido é previsto na Resolução Normativa nº 95/CUn/2017, Art. 55, § 2, de 4 de abril de 2017, e exigido para teses e dissertações escritas em idiomas estrangeiros (com exceção dos cursos pertinentes ao estudo de idiomas estrangeiros – Programa de Pós-Graduação em Estudos da Tradução e Programa de Pós-Graduação em Inglês: Estudos Linguísticos e Literários).
O resumo expandido é considerado um elemento pré-textual e deverá ser incluído no trabalho após o resumo e antes do abstract. Deverá iniciar em página impar (no anverso de uma folha) continuando no verso da folha. O texto deverá seguir o formato A5, com margens espelhadas: superior 2,0 cm, inferior 1,5 cm, interna 2,5 cm e externa 1,5. Deve ser empregada a fonte Time New Roman.  Todo o texto deve ser digitado em tamanho 10,5. O espaçamento entre as linhas deverá ser simples. A expressão “resumo expandido” deve seguir a mesma tipografia das demais sessões primárias do trabalho.
O texto do resumo expandido deve ser redigido em português e conter as seguintes seções (ver modelo): Introdução, Objetivos, Metodologia, Resultados e Discussão e Considerações Finais. Deve apresentar no mínimo duas (02) e, no máximo, cinco (05) páginas contendo a mesma formatação em A5 do resumo e do abstract, bem como palavras-chave.

\tituloresumoexpandido{Objetivos} 
Lorem ipsum dolor sit amet, consectetur adipiscing elit. Phasellus vitae dolor lacus. Ut accumsan vitae felis nec porttitor. Integer interdum fringilla feugiat. Nullam pulvinar sit amet tellus eget maximus. Donec sit amet magna eget justo semper fermentum vel eget velit. In iaculis imperdiet mauris, ac ornare libero placerat non. Nulla libero lectus, ullamcorper ac ornare eget, pulvinar ac nulla. Curabitur vestibulum non nisl eget sagittis. Proin gravida lacus id eros bibendum interdum. Mauris ullamcorper elementum tortor sed consequat. Integer tempus, est a lobortis vehicula, nisi mi fringilla augue, non semper leo metus in quam.
 
\tituloresumoexpandido{Metodologia}
Lorem ipsum dolor sit amet, consectetur adipiscing elit. Phasellus vitae dolor lacus. Ut accumsan vitae felis nec porttitor. Integer interdum fringilla feugiat. Nullam pulvinar sit amet tellus eget maximus. Donec sit amet magna eget justo semper fermentum vel eget velit. In iaculis imperdiet mauris, ac ornare libero placerat non. Nulla libero lectus, ullamcorper ac ornare eget, pulvinar ac nulla. Curabitur vestibulum non nisl eget sagittis. Proin gravida lacus id eros bibendum interdum. Mauris ullamcorper elementum tortor sed consequat. Integer tempus, est a lobortis vehicula, nisi mi fringilla augue, non semper leo metus in quam.
 
\tituloresumoexpandido{Resultados e Discussão}
Lorem ipsum dolor sit amet, consectetur adipiscing elit. Phasellus vitae dolor lacus. Ut accumsan vitae felis nec porttitor. Integer interdum fringilla feugiat. Nullam pulvinar sit amet tellus eget maximus. Donec sit amet magna eget justo semper fermentum vel eget velit. In iaculis imperdiet mauris, ac ornare libero placerat non. Nulla libero lectus, ullamcorper ac ornare eget, pulvinar ac nulla. Curabitur vestibulum non nisl eget sagittis. Proin gravida lacus id eros bibendum interdum. Mauris ullamcorper elementum tortor sed consequat. Integer tempus, est a lobortis vehicula, nisi mi fringilla augue, non semper leo metus in quam.

\tituloresumoexpandido{Considerações Finais}
Lorem ipsum dolor sit amet, consectetur adipiscing elit. Phasellus vitae dolor lacus. Ut accumsan vitae felis nec porttitor. Integer interdum fringilla feugiat. Nullam pulvinar sit amet tellus eget maximus. Donec sit amet magna eget justo semper fermentum vel eget velit. In iaculis imperdiet mauris, ac ornare libero placerat non. Nulla libero lectus, ullamcorper ac ornare eget, pulvinar ac nulla. Curabitur vestibulum non nisl eget sagittis. Proin gravida lacus id eros bibendum interdum. Mauris ullamcorper elementum tortor sed consequat. Integer tempus, est a lobortis vehicula, nisi mi fringilla augue, non semper leo metus in quam.}

\textabstract {Resumo traduzido para outros idiomas, neste caso, inglês. Segue o formato do resumo feito na língua vernácula. As palavras-chave traduzidas, versão em língua estrangeira, são colocadas abaixo do texto precedidas pela expressão ``Keywords'', separadas por ponto.}
\keywords{Keyword 1. Keyword 2. Keyword 3.}
